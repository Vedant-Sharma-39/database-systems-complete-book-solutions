\documentclass[../../main.tex]{subfiles}

\begin{document}

\subsection*{3.2 Rules About Functional Dependencies}

\subsubsection*{Exercise 2.1}

a)

First, we need to find all the closures of the subsets
of the attributes.

\begin{align*}
  \{A\}^{+} &= \{A\},   \{B\}^{+} = \{B\},
  \{C\}^{+} = \{A, C, D\}, \{D\}^{+}= \{A,D\} \\
  \{AB\}^{+} &= \{A,B,C,D\},  \{AC\}^{+} = \{A,C,D\},
  \{AD\}^{+} = \{A,D\} \\
  \{BC\}^{+} &= \{A,B,C,D\}, \{BD\}^{+} = \{A,B,C,D\},
  \{CD\}^{+} = \{A,C,D\} \\
  \{ABC\}^{+} &= \{A,B,D\}^{+} = \{BCD\}^{+} = \{A,B,C,D\},
  \{ACD\}^{+} = \{A,C,D\}
\end{align*}

The answer is obvious.

b)

From a), it is easy to get that the keys are:

$$
AB, BC \ \mathbf{AND} \ BD
$$

c)

The superkeys are all those that contain one of those three
keys. The proper superkeys are:

$$
ABC, ABD, BCD \ \mathbf{AND} \ ABCD
$$

\subsubsection*{Exercise 2.2}

i)

It is obvious that $\{A\}^{+} = \{A,B,C,D\}$. So there is no
need to calculate the closures of subsets which contain $A$.
Use this idea, we can simplify the calculation.

\begin{align*}
  \{A \_ \ \_ \ \_ \}^{+} &= \{A,B,C,D\} \\
  \{B \ \_ \ \_ \}^{+} &= \{B,C,D\} \\
  \{C\}^{+} &= \{C\}, \{D\}^{+} = \{D\}, \{CD\}^{+} = \{C,D\}
\end{align*}

The keys are:

$$
A, AB, AC \ \mathbf{AND} \ AD
$$

The superkeys are:

$$
ABC, ABD, ACD \ \mathbf{AND} \ ABCD
$$

ii)

\begin{align*}
  \{A\}^{+} &= \{A\},   \{B\}^{+} = \{B\},
  \{C\}^{+} = \{C\}, \{D\}^{+}= \{D\} \\
  \{AB\}^{+} &= \{A,B,C,D\},  \{AC\}^{+} = \{A,C\},
  \{AD\}^{+} = \{A,B,C,D\} \\
  \{BC\}^{+} &= \{A,B,C,D\}, \{BD\}^{+} = \{B,D\},
  \{CD\}^{+} = \{A,B,C,D\} \\
  \{ABC\}^{+} &= \{A,B,D\}^{+} = \{BCD\}^{+} = \{ACD\}^{+} = \{A,B,C,D\}
\end{align*}

The keys are:

$$
AB, AD, BC \ \mathbf{AND} \ CD
$$

The superkeys are:

$$
ABC, ABD, BCD, ACD \ \mathbf{AND} \ ABCD
$$


iii)

\begin{align*}
  \{A \_ \ \_ \ \_ \}^{+} &= \{A,B,C,D\} \\
  \{B\_ \ \_ \ \_ \}^{+} &= \{A,B,C,D\} \\
  \{C\_ \ \_ \ \_ \}^{+} &= \{A,B,C,D\} \\
  \{D\_ \ \_ \ \_ \}^{+} &= \{A,B,C,D\} \\
\end{align*}

The keys are:

$$
A, B, C \ \mathbf{AND} \ D
$$

The superkeys are:

$$
\mbox{Subset} \{A,B,C,D\} - \mbox{keys}
$$

\subsubsection*{Exercise 2.3}

a)

$$
\{A_{1}, A_{2}, \cdots, A_{n}\} ^ {+} = \{A_{1}, A_{2}, \cdots, A_{n}, B\}
$$

And we can easily get:

$$
\{A_{1}, A_{2}, \cdots, A_{n}, C\} ^ {+} = \{A_{1}, A_{2}, \cdots, A_{n}, B\}
$$

Thus proved.

b)

It can be easily proved by trivial FD.

c)

$$
C_{1}C_{2}\cdots C_{k} \to D \Rightarrow B_{1}
B_{2}\cdots B_{m} E_{1}E_{2}\cdots E_{j}
$$

Use the augmentation rule.

$$
A_{1}A_{2}\cdots A_{n} \to B_{1}B_{2} \cdots B_{m}
\Rightarrow A_{1}A_{2}\cdots A_{n}E_{1}E_{2}\cdots E_{j}
\to B_{1}B_{2}\cdots B_{m} E_{1}E_{2}\cdots E_{j}
$$

Use the transition rule.

$$
A_{1}A_{2}\cdots A_{n}E_{1}E_{2}\cdots E_{j} \to D
$$

d)

$$
\{A_{1}, A_{2}, \cdots, A_{n}\} ^ {+} = \{A_{1}, A_{2}, \cdots, A_{n}, B_{1}
,B_{2} \cdots B_{m} \}
$$

$$
\{C_{1}, C_{2}, \cdots, C_{k}\} ^ {+} = \{C_{1}, C_{2}, \cdots, C_{k}, D_{1}
,D_{2} \cdots D_{j} \}
$$

Using augmentation and combing rules, thus proved.

\subsubsection*{Exercise 2.4}

a)

Attribute $A$ represents Social Security Number and $B$
represented a person's name.

b)

Attribute $A$ represents Social Security Number and $B$
represented a person's gender, $C$ represents a person's name.

c)

Attribute $A$ represents latitude and $B$ represents longitude,
$C$ represents a point on the world map.

\subsubsection*{Exercise 2.5}

Consider the relationship $R(A,B,C,D)$. Suppose there exists
a non-trivial dependency $A \to B$. Then:

\begin{itemize}
  \item There exist an attribute $B$ that can be functionally
        determined by all other attributes.
  \item For some value of $A$ the value of $B$ can be functionally
        determined
\end{itemize}

It is given that no attribute can be functionally determined by all
other attributes. This is contradiction.

\subsubsection*{Exercise 2.6}

First, we convert $X \subseteq Y$ to $Y = X + S$. It is obvious that
$\{Y\}^{+} = \{X\}^{+} + \{S\}^{+}$. Thus

$$
X^+ \subseteq Y^+
$$

\subsubsection*{Exercise 2.7}

We prove it by concept, when we find the closure of $X$, it must go
to the end. So for $(X^+)^+$, it can't find any suitable
attributes.

\subsubsection*{Exercise 2.8}

a)

\begin{align*}
  A &\to A \\
  B &\to B \\
  C &\to C \\
  D &\to D
\end{align*}

b)

$$
ABC \to D
$$

c)

\begin{align*}
  A &\to B \\
  ABC &\to D
\end{align*}

\subsubsection*{Exercise 2.9}

\begin{itemize}
  \item $A \to B, B \to A, B \to C , C \to B$
  \item $A \to B, B \to C, C \to A$
  \item $B \to A, A \to C, C \to B$
  \item $C \to A, A \to B, B \to C$
  \item $A \to C, B \to A, C \to A, C \to B$
  \item $C \to A, A \to B, B \to A, B \to C$
  \item $C \to B, C \to A, B \to C, A \to B$
  \item $B \to C, B \to A, A \to B, C \to B$
  \item $A \to B, A \to C, B \to A, C \to B$
\end{itemize}

\subsubsection*{Exercise 2.10}

a)

\begin{align*}
  \{A\}^+ &= \{A\}, \{B\}^+ = \{B\}, \{C\}^+ = \{A, C, E\} \\
  \{AB\}^+ &= \{A, B, C, D ,E\}, \{BC\}^+ = \{A, B, C, E\} \\
  \{AC\}^+ &= \{A, C, E\}, \{ABC\}^+ = \{A, B, C, D,E\}
\end{align*}

So we can find the FD in $S$: $AB \to C$ and $C \to A$.

b)

\begin{align*}
  \{A\}^+ &= \{A,D\}, \{B\}^+ = \{B\}, \{C\}^+ = \{C\} \\
  \{AB\}^+ &= \{A, B, D, E\}, \{BC\}^+ = \{B, C\} \\
  \{AC\}^+ &= \{A, C, E\}, \{ABC\}^+ = \{A, B, C, D, E\}
\end{align*}

From the above closure, $A$ and $C$ are the only attributes
presents in $S$. So we can find FD in $S$: $A \to B, C \to B$.

c)

\begin{align*}
  \{A\}^+ &= \{A\}, \{B\}^+ = \{B\}, \{C\}^+ = \{C\} \\
  \{AB\}^+ &= \{A, B, D\}, \{BC\}^+ = \{A, B, C, D, E\} \\
  \{AC\}^+ &= \{A, B, C, D,E\}, \{ABC\}^+ = \{A, B, C, D, E\}
\end{align*}

For $\{AB\}^+$, $A$ and $B$ are the only attributes
presents in $S$.For $\{AC\}^+$ and $\{BC\}^+$, $A$, $B$ and $C$
are the only attributes. So we can find FD in $S$:
$A \to B, B \to A, C \to B, C \to A$.

d)

\begin{align*}
  \{A\_ \ \_\}^+ &= \{A,B,C\} \\
  \{B\_ \ \_\}^+ &= \{A,B,C\} \\
  \{C\_ \ \_\}^+ &= \{A,B,C\}
\end{align*}

So we can find FD in $S$:
$A \to B, B \to C, C \to A$.

\subsubsection*{Exercise 2.11}

We have relation schema $S(A,B,C,D)$ with

\begin{align*}
  A &\to B \\
  B &\to C \\
  B &\to D
\end{align*}

We can compute $\{A\}^+ = \{A,B,C,D\}$. So we can get

\begin{align*}
  A &\to B \\
  B &\to C
\end{align*}

Gives $A \to C$. Thus, if FD $F$ follows from the given FD's,
then $F$ can be proved from the given FD's using Armstrong's
axioms.

\end{document}
