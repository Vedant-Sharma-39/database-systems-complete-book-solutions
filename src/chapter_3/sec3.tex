\documentclass[../../main.tex]{subfiles}

\begin{document}

\subsection{3.3 Design of Relational Database Schemas}

\subsubsection*{Exercise 3.1}

a)

\begin{align*}
  \{AB\}^{+} &= \{A,B,C,D\} \\
  \{C\}^{+} &= \{A,C,D\} \\
  \{D\}^{+} &= \{A,D\}
\end{align*}

The second and the third FD's violate the BCNF.

By choosing FD $C \to D$, we split the relation into
$R_{1}(A,C,D)$ and $R_{2}(B,C)$

Next, we need to find FD's in $R_{1}$ and $R_{2}$. There are
only two attributes in $R_{2}$, so $R_{2}$ is in BCNF.

\begin{align*}
  \{A\}^+ &= \{A\}, \{C\}^+ = \{A, C, D\}, \{D\}^+ = \{A,D\} \\
  \{AC\}^+ &= \{A, C, D\}, \{AD\}^+ = \{A, D\} \\
  \{CD\}^+ &= \{A, C, D\}
\end{align*}

So we can find the FD's in $R_{1}$ are: $C \to D, D \to A$. $C$
is not a superkey. So again, dut to the FD $D \to A$,we compose
$R_{1}$ into $R_{3}(A,D)$ and $R_{4}(C,D)$.

b)

We simply compute $\{B\}^{+} = \{B,C,D\}$. Since $A$ is not in the
closure, there is a violation of the BCNF.

By choosing FD $B \to C$, we split the relation into
$R_{1}(B,C,D)$ and $R_{2}(A,B)$.

Next, we need to find FD's in $R_{1}$ and $R_{2}$. There are
only two attributes in $R_{2}$, so $R_{2}$ is in BCNF.

\begin{align*}
  \{B\}^{+} &= \{B,C,D\} \\
  \{C\}^{+} &= \{C\} \\
  \{D\}^{+} &= \{D\} \\
  \{CD\}^{+} &= \{C,D\}
\end{align*}

So we can find the FD's in $R_{1}$ are: $B \to C$ and $B \to D$.

c)

\begin{align*}
  \{AB\}^{+} &= \{A,B,C,D\} \\
  \{BC\}^{+} &= \{A,B,C,D\} \\
  \{CD\}^{+} &= \{A,B,C,D\} \\
  \{AD\}^{+} &= \{A,B,C,D\}
\end{align*}

No BCNF violations, Done.

d)

\begin{align*}
  \{A\}^{+} &= \{A,B,C,D\} \\
  \{B\}^{+} &= \{A,B,C,D\} \\
  \{C\}^{+} &= \{A,B,C,D\} \\
  \{D\}^{+} &= \{A,B,C,D\}
\end{align*}

No BCNF violations, Done.

e)

\begin{align*}
  \{AB\}^{+} &= \{A,B,C,D\} \\
  \{DE\}^{+} &= \{B,C,D,E\} \\
  \{B\}^{+}  &= \{B,D\}
\end{align*}

All three FD's violate BCNF.

By choosing FD $AB \to C$, we split the relation into
$R_{1}(A,B,C,D)$ and $R_{2}(A,B,E)$.

For $R_{2}$:

\begin{align*}
  \{A\}^+ &= \{A\}, \{B\}^+ = \{B\}, \{E\}^+ = \{E\} \\
  \{AB\}^+ &= \{A, B\}, \{BE\}^+ = \{B,E\} \\
  \{AE\}^+ &= \{A, E\}, \{ABE\}^+ = \{A, B, E\}
\end{align*}

$R_{2}$ doesn't violate BCNF.

For $R_{1}$:

\begin{align*}
  \{A\}^+ &= \{A\}, \{B\}^+ = \{B, D\},
  \{C\}^+ = \{C\}, \{D\}^+ = \{D\} \\
  \{AB\}^+ &= \{A,B,C,D\}
\end{align*}

So we can find the FD's in $R_{1}$ are: $AB \to C$,$AB \to D$
and $B \to D$.

Since $B \to D$ violates BCNF. So we again decompose $R_{1}$ into
$R_{3}(B,D)$ and $R_{4}(A,B,C)$. Done.

f)

For simplicity, we only give one answer here:

$$
R_{1}(A), R_{2}(A,B), R_{3}(A,C), R_{4}(A,D), R_{5}(D,E)
$$

\subsubsection*{Exercise 3.2}

For $A \to B$ and $A \to C$:

$$
\{A\}^+ = {A,B,C}
$$

We decompose $R$ into $R_{1}(A,B,C)$ and $R_{2}(A,D)$

For $A \to BC$, the $\{A\}^{+}$ won't be changed. So we
can get the same result.

\subsubsection*{Exercise 3.3}

Well, from Exercise 3.2, we can know that
$\{A\}^+$ will never be changed.So we
can get the same result.

\subsubsection*{Exercise 3.4}

  \begin{table}[H]
    \centering
    \begin{tabular}{@{}ccc@{}}
    \toprule
    \textbf{Factory} & \textbf{Model} & \textbf{Year} \\ \midrule
    Nokia            & 3310           & 2000          \\
    Samsung          & s3             & 2012          \\
    Samsung          & s20            & 2020          \\ \bottomrule
    \end{tabular}
  \end{table}

\end{document}
