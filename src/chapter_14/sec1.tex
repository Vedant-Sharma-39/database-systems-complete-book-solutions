\documentclass[../../main.tex]{subfiles}

\begin{document}

\subsection{14.1 Index-Structure Basics}

\subsubsection*{Exercise 1.1}

a)

The number of blocks to store the record is:

$$
\frac{n}{3}
$$

The number of blocks to store the pointer is:

$$
\frac{n}{10}
$$

Thus, the answer is:

$$
\frac{n}{3} + \frac{n}{10} = \frac{13n}{30}
$$

b)

The number of blocks to store the record is:

$$
\frac{n}{3}
$$

The number of blocks to store the pointer is:

$$
\frac{n / 3}{10}
$$

Thus, the answer is:

$$
\frac{n}{3} + \frac{n / 3}{10} = \frac{11n}{30}
$$

\subsubsection*{Exercise 1.2}

a)

The number of blocks to store the record is:

$$
\frac{n}{200 \times 80\%}
$$

The number of blocks to store the pointer is:

$$
\frac{n}{30 \times 80\%}
$$

Thus, the answer is:

$$
\frac{n}{200 \times 80\%} + \frac{n}{30 \times 80\%} = \frac{23n}{480}
$$

b)

The number of blocks to store the record is:

$$
\frac{n}{30 \times 80\%}
$$

The number of blocks to store the pointer is:

$$
\frac{n}{30 \times 80\% \times 200 \times 80\%}
$$

Thus, the answer is:

$$
\frac{n}{30 \times 80\%} + \frac{n}{30 \times 80\% \times 200 \times 80\%} = \frac{161n}{3840}
$$

\end{document}
