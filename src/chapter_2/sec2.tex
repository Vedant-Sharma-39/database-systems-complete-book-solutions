\subsection*{2.2 Basics of the Relational Model}

\subsubsection*{Exercise 2.1}

a)

Attributes of relation \verb|Accounts| are \verb|acctNo|
, \verb|type| and \verb|balance|.

Attributes of relation \verb|Customers| are \verb|firstName|
, \verb|lastName|, \verb|idNo| and \verb|account|.

b)

For the relation \verb|Accounts|:

\begin{itemize}
  \item (12345, savings, 12000)
  \item (23456, checking, 1000)
  \item (34567, savings, 25)
\end{itemize}

For the relation \verb|Customers|:

\begin{itemize}
  \item (Robbie, Banks, 901-222 12345)
  \item (Lena, Hand, 805-333, 12345)
  \item (Lena, Hand, 805-333, 23456)
\end{itemize}

c)

\begin{itemize}
  \item (12345, savings, 12000)
  \item (Robbie, Banks, 901-222 12345)
\end{itemize}

d)

\begin{itemize}
  \item Accounts(acctNo, type, balance)
  \item Customers(firstName, lastName, idNo, account)
\end{itemize}

e)

\begin{itemize}
  \item Accounts(acctNo: integer, type: string, balance: integer)
  \item Customers(firstName: string, lastName:string,
                  idNo: string, account: integer)
\end{itemize}

f)

\begin{itemize}
  \item Accounts(integer, string, integer)
  \item Customers(string, string, string, integer)
\end{itemize}

g)

Just swap the attributes (whatever you like).

\begin{itemize}
  \item Accounts(type, accNo, balance)
  \item Customers(lastName, firstName, idNo, account)
\end{itemize}

\subsubsection*{Exercise 2.2}

Student(id, firstName, lastName)

\verb|id| is the key.

\subsubsection*{Exercise 2.3}

It is a combinatorial problem. It is important how many
attributes we have. Order of tuples doesn't matter.

a) $3 \times 2 \time 1 = 6$

b) $4 \times 3 \times 2 \times 1 = 12$

c) $n!$
