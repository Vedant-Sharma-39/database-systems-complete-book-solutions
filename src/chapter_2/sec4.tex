\subsection*{2.4 An Algebraic Query Language}

\subsubsection*{Exercise 4.1}

a)

\begin{align*}
  R1 &:= \sigma_{speed \geq 3.00}(PC) \\
  R2 &:= \pi_{model}(R1)
\end{align*}

b)

\begin{align*}
  R1 &:= \sigma_{hd \geq 100}(Laptop) \\
  R2 &:= Product \bowtie (R2) \\
  R3 &:= \pi_{maker}(R3)
\end{align*}

c)

\begin{align*}
  R1 &:= \sigma_{maker=B}(Product \bowtie PC) \\
  R2 &:= \sigma_{maker=B}(Product \bowtie Laptop) \\
  R3 &:= \sigma_{maker=B}(Product \bowtie Printer) \\
  R4 &:= \pi_{model, price}(R1) \cup \pi_{model, price}(R2)
         \cup \pi_{model, price}(R3)
\end{align*}

d)

\begin{align*}
  R1 &:= \sigma_{color=true AND type=layser}(Printer) \\
  R2 &:= \pi_{model}(R1)
\end{align*}

e)

\begin{align*}
  R1 &:= \sigma_{type=laptop}(Product) \\
  R2 &:= \sigma_{type=pc}(Product) \\
  R3 &:= \pi_{maker}(R1) - \pi_{maker}(R2)
\end{align*}

f)

\begin{align*}
  R1 &:= \rho_{PC1}(PC) \\
  R2 &:= \rho_{PC2}(PC) \\
  R3 &:= R1 \bowtie_{PC1.hd = PC2.hd AND PC1.model <> PC2.model} R2 \\
  R4 &:= \pi_{hd}(R3)
\end{align*}

g)

The most important thing here is to notice that \emph{a pair
should be listed only once}. Here we use \verb|PC1.model < PC.model|
to achieve this functionality.

\begin{align*}
  R1 &:= \rho_{PC1}(PC) \\
  R2 &:= \rho_{PC2}(PC) \\
  R3 &:= R1 \bowtie_{PC1.speed = PC2.speed AND PC1.ram =
         P2.ram and PC1.model < PC2.model} R2
\end{align*}

h)

\begin{align*}
  R1 &:= \pi_{model}(\sigma_{speed \geq 2.80}(PC))
  \cup \pi_{model}(\sigma_{speed \geq 2.80}(PC)) \\
  R2 &:= \pi_{maker, model}(R1 \bowtie Product) \\
  R3 &:= \rho_{R3(maker2, model2)}(R2) \\
  R4 &:= R2 \bowtie_{maker = maker2 AND model <> model2} R3 \\
  R5 &:= \pi_{maker}(R4)
\end{align*}

i)

\begin{align*}
  R1 &:= \pi_{model, speed}(PC) \\
  R2 &:= \pi_{model, speed}(PC) \\
  R3 &:= R1 \cup R2 \\
  R4 &:= \rho_{R4(model2, speed2)}(R3) \\
  R5 &:= \pi_{model, speed}(R3 \bowtie_{speed < speed2} R4) \\
  R6 &:= R3 - R5 \\
  R7 &:= \pi_{maker}(R6 \bowtie Product)
\end{align*}

j)

\begin{align*}
  R1 &:= \pi_{maker, speed}(PC \bowtie Product) \\
  R2 &:= \rho_{R2(maker2, speed2)}(R1) \\
  R3 &:= \rho_{R3(maker3, speed3)}(R1) \\
  R4 &:= R1 \bowtie_{maker = maker2 AND speed <> speed2} R2 \\
  R5 &:= R4 \bowtie_{maker = maker3 AND speed3 <> speed2
         AND speed3 <> speed} R3 \\
  R6 &:= \pi_{maker}(R5)
\end{align*}

k)

\begin{align*}
  R1 &:= \pi_{maker, model}(PC \bowtie Product) \\
  R2 &:= \rho_{R2(maker2, model2)}(R1) \\
  R3 &:= \rho_{R3(maker3, model3)}(R1) \\
  R4 &:= \rho_{R4(maker4, model4)}(R1) \\
  R5 &:= R1 \bowtie_{maker=maker2 AND model <> model2} R2 \\
  R6 &:= R3 \bowtie_{maker=maker3 AND model3 <> model2
         AND model3 <> model}R5 \\
  R7 &:= R4 \bowtie_{maker4=maker AND (model4 = model OR
         model4 = model2 OR model4 = model3)} R6 \\
  R8 &:= \pi_{maker}(R7)
\end{align*}

\subsubsection*{Exercise 4.2}

It's a dirty job. I omit detail here. But do remember
\emph{a query tree is a tree data structure representing
a relational algebra expression}.

\subsubsection*{Exercise 4.3}

a)

\begin{align*}
  R1 &:= \sigma_{bore \geq 16}(Classes) \\
  R2 &:= \pi_{class,country}(R1)
\end{align*}

b)

\begin{align*}
  R1 &:= \sigma_{launched < 1921}(Ships) \\
  R2 &:= \pi_{name}(R1)
\end{align*}

c)

\begin{align*}
  R1 &:= \sigma_{battle=DenmarkStrait AND result=sunk}
         (Outcomes) \\
  R2 &:= \pi_{ship}(R1)
\end{align*}

d)

\begin{align*}
  R1 &:= Classes \bowtie_{launched > 1921 AND
         displacement > 35000} Ships \\
  R2 &:= \pi_{name}(R1)
\end{align*}

e)

\begin{align*}
  R1 &:= \sigma_{battle = Guadalcanal}(Outcomes) \\
  R2 &:= Ships \bowtie_{ship = name} R1 \\
  R3 &:= Classes \bowtie R2 \\
  R4 &:= \pi_{name, displacement, numGuns}(R3)
\end{align*}

f)

\begin{align*}
  R1 &:= \pi_{ship}(Outcomes) \\
  R2 &:= \rho_{R2(name)}(R1) \\
  R3 &:= \pi_{name}(Ships \cup R2)
\end{align*}

g)

\begin{align*}
  R1 &:= \pi_{class}(Classes) \\
  R2 &:= \pi_{class}(\sigma_{class <> name}(Ships)) \\
  R3 &:= R1 - R2
\end{align*}

h)

\begin{align*}
  R1 &:= \pi_{country}(\sigma_{type=bb}(Classes)) \\
  R2 &:= \pi_{country}(\sigma_{type=bc}(Classes)) \\
  R3 &:= R1 \cap R2
\end{align*}

i)

\begin{align*}
  R1 &:= \pi_{ship, result, date}(Battles
         \bowtie_{battle = name} Outcomes) \\
  R2 &:= \rho(R2(ship, result, date))(R1) \\
  R3 &:= \pi_{ship}(R1 \bowtie_{ship = ship2 AND
         result = danamaged AND date < date2} R2)
\end{align*}

\subsubsection*{Exercise 4.4}

Just like Exercise 4.2, I omit detail here.

\subsubsection*{Exercise 4.5}

Natural join does not use any comparison operator. It
does not concatenate the way a Cartesian product does.
We can perform a Natural Join only if there is at least
one common attribute that exists between two relations.
In addition, the attributes must have the same name and
domain.

\subsubsection*{Exercise 4.6}

Here we assume that there are two relations:

\begin{itemize}
  \item $A(a_{1},a_{2},\dots,a_{n})$
  \item $B(b_{1},b_{2},\dots,b_{n})$
\end{itemize}

First, the $\cap$ operation is \emph{monotone}. Assume that
$C = A \cap B$. When $A$ adds a tuple $a_{n + 1}$, $C$ may
be not changed or become $C' = C \cup a_{n + 1}$ such that
$C \subseteq C'$. For other situations, the idea is the same.

Second, the $\sigma$ operator is \emph{monotone}. Assume that
$C = \sigma_{c}(A)$. When $A$ adds a tuple $a_{n + 1}$, $C$ may
be not changed or become $C' = C \cup a_{n + 1}$ such that
$C \subseteq C'$.

\subsubsection*{Exercise 4.7}

a)

\begin{itemize}
  \item $\max =  n + m$
  \item $\min = \max(n, m)$
\end{itemize}

b)

\begin{itemize}
  \item $\max = n \times m$
  \item $\min = 0$
\end{itemize}

c)

\begin{itemize}
  \item $\max = n \times m$
  \item $\max = 0$
\end{itemize}

d)

\begin{itemize}
  \item $\max = n$
  \item $\min = 0$
\end{itemize}

\subsubsection*{Exercise 4.8}

\begin{itemize}
  \item $\pi_{1\dots n}(R \bowtie S)$
  \item $\pi_{1\dots n}(\sigma_{R.1=S.1 
        \dots R.K=S.K}(R \times S))$
  \item $R \bowtie_{1, \dots k}(S)$
\end{itemize}

\subsubsection*{Exercise 4.9}

$R - \pi_{1,\dots,n}(R \bowtie S)$

\subsubsection*{Exercise 4.10}

An intuitive property of the division operator of
the relational algebra is simply that it is the inverse
of the cartesian product.
