\documentclass[../../main.tex]{subfiles}

\begin{document}

\subsection{13.3 Accelerating Access to Secondary Storage}

\subsubsection*{Exercise 3.1}

From the \hyperlink{Section13.2}{Section 13.2}, we can know that the average latency
and transfer time is $3.8$ and $0.5$

a)

\begin{itemize}
  \item Request 1: cylinder 8000, since the disk head already on cylinder 8000 there
        is no seek time. The access time is $4.3$ms. And the completion time is $4.3$ms
  \item Request 2: cylinder 48000, the seek time is $ 1 + (0.00025 \times 40000) = 11$ms,
        And the completion time is $11 + 4.3 + 4.3 = 19.6$ms
  \item Request 3: cylinder 40000, the seek time is $ 1 + (0.00025 \times 8000) = 3$ms
        And the completion time is $3 + 4.3 + 19.6 = 26.9$ms
  \item Request 4: cylinder 4000, the seek time is $1 + (0.00025 \times 36000) = 10$ms
        And the completion time is $10 + 4.3 + 26.9 = 41.3$ms
\end{itemize}

b)

We need to only calculate the seek time.

\begin{itemize}
  \item Request 1: cylinder 8000, since the disk head already on cylinder 8000 there
        is no seek time. The completion time is $4.3$ms
  \item Request 2: cylinder 48000, the seek time is $ 1 + (0.00025 \times 40000) = 11$ms
        And the completion time is $11 + 4.3 + 4.3 = 19.6$ms
  \item Request 3: cylinder 4000, the seek time is $ 1 + (0.00025 \times 44000) = 12$ms
  \item And the completion time is $12 + 4.3 + 19.6 = 35.9$ms
  \item Request 4: cylinder 40000, the seek time is $1 + (0.00025 \times 36000) = 10$ms
        And the completion time is $10 + 4.3 + 35.9 = 50.2$ms
\end{itemize}

\subsubsection*{Exercise 3.2}

We use the data from the \hyperlink{Section13.2}{Section 13.2}

a)

Recall that the Megatron 747 has a transfer time (in milliseconds) of 0.25,
and an average rotational latency of 4.17. If the average seek
moves 1/3 of the way across half of the 16,384 tracks, the average seek time for the
Megatron 747 is $1 + .001 \times (16384/6) = 3.73.1+.001 \times (16384/6)=3.73.$
Thus, the answer is one block per $0.25 + 4.17 + 3.73 = 8.15$ ms, or 123 blocks per second,
on each disk. Thus, the system can read 246 blocks per second.

b)

For a "regular" Megatron 747, the average latency is $0.25 + 4.17 + 6.46 = 10.880.25+4.17+6.46=10.88$ ms.
However, if we have two, mirrored disks, each can be handling a request at the same time, or
one read per $5.44$ milliseconds. That gives us $184$ blocks per second.

c)

Restricting each disk to half the tracks means that
if there is not always a queue of waiting requests (which slows the applications
that the disk is supporting), then there might be two requests
for the same half of the disk, in which case one disk is idle and a request waits.

\subsubsection*{Exercise 3.3}

a)

Number of request we can say that will be $n$; Head travels across $65536$
takes $n+16.38$ ms. The result in the rotational latency of $4.17n$ ms.
The time required for transfer is $0.13$ ms which in turn gives the total
transfer of $0.13n$ ms

\begin{align*}
&n + 16.38 + 4.17n + 0.13n = An \\
&n = \frac{16.38}{A - 5.3}
\end{align*}

b)

$$
\frac{16.38}{A - 5.3}
$$

c)

The waiting time for the request will be 0 when the request comes in right.
In some cases, the request comes only after the pass has started
which is a worst case.

When the request gets its pass then the wait must get serviced. The wait
time for service would be 0. The request will wait for the service
at 0 ms, in the best case.

$$
\frac{16.38}{A - 5.3} - 0.065 \ \mathbf{ms}
$$

\subsubsection*{Exercise 3.4}

Think of the requests as a random sequence of the integers $1$ through $4$.
This sequence can be divided up into segments that do not contain two of
the same integer, in a greedy way. For example, the sequence $123142342431$
would be divided into $123$, $1423$, $42$, and $31$. The disks are serving all
the requests from one segment, and each request is generated and starts at
about the same time. When a segment is finished, the waiting request begins,
along with other requests for other disks that are, by our assumption,
generated almost immediately and thus finish at about the same time.

The question is thus: if I choose numbers $1$, $2$, $3$, and $4$ at random, how many choices, on the average,
can I make without a repeat? The cases are:

\begin{itemize}
  \item $1 / 4$ of the time we get an immediate repeat on the second choice, and the length is therefore $1$.
  \item $(3/4)(1/2)=3/8$ of the time we get our first repeat at third number, and the length is $2$.
  \item $(3/4)(1/2)(3/4)=9/32$ of the time we get our first repeat at the fourth number, and our length is $3$.
  \item The remaining fraction of the time, $3/32$, we get four different numbers, and our length is $4$.
\end{itemize}

The average length is therefore:

$$
(1 / 4) \times 1 + (3 / 8) \times 2 + (9 / 32) \times 3 + (3 / 32) \times 4= 71 / 32
$$

\subsubsection*{Exercise 3.5}

The disk serves all requests from one segment and each
of these requests are generated and started at the same time.
The waiting request begins when the segment is finished. This
request is raised along with other disks and finishes at the
same time. We have a request of the random sequence of integers
1 through k. Now we can choose the numbers at random to find
the number of choices without repeat.

\begin{itemize}
  \item When the length is 1, the it takes $\frac{1}{k}$ of the time
        for the immediate repeat on the second choice.
  \item When the length is 2, the time will be $\frac{1}{k} \times 2$.
  \item When the length is $k$, the time will be $\frac{1}{k} \times k$
\end{itemize}

Average length is:

\begin{align*}
  &\frac{1}{k} + 2 \times \frac{1}{k} + \cdots + k \times \frac{1}{k} \\
  &= (1 + 2 + \cdots + k) \frac{1}{k} \\
  &= \frac{k \times (k + 1)}{2} \frac{1}{k} \\
  &= \frac{k}{2}
\end{align*}

\end{document}
