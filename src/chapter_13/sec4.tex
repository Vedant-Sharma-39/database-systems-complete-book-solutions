\documentclass[../../main.tex]{subfiles}

\begin{document}

\subsection{13.4 Disk Failures}

\subsubsection*{Exercise 4.1}

a)

There is the odd number of 1's so the parity bit is 1.

b)

There is the even number of 1's so the parity bit is 1

c)

There is an odd number of 1's, so the parity bit is 1

\subsubsection*{Exercise 4.2}

a) 1 0

b) 0 0

c) 1 0

\subsubsection*{Exercise 4.3}

$$
\frac{1}{1095} \times 0.04 = 3.65 \ \mathbf{ms}
$$

\subsubsection*{Exercise 4.4}

a)

To compute the mean time to failure, we can compute the probability that the system fails in a given year. The MTTF will be the
inverse of that probability. Note that there are 8760 hours in a year. The system fails if the second disk fails while the first
is being repaired. The probability of a failure of one of the two disks in a year is $2F$.
The probability that the second disk will fail during the H hours that the other is being prepared is
$FH/ 8760$. Thus, the probability of a failure in any year is $2F^{2H/8760}$, and the MTTF is
$4380 / F^{2H}$.

b)

The system fails if any of the other $N-1$ disks fails while the first is being repaired.
The probability of a failure of one of the $N$ disks in a year is $NF$. The probability
that a second disk will fail during the $H$ hours that the other is being prepared
is $(N - 1)FH / 8760$. Thus, the probability of a failure in any year is $N(N-1)F^{2H/8760}$.

\subsubsection*{Exercise 4.5}

$$
\frac{8760^{2}}{3F^{3}H^{2}}
$$

\subsubsection*{Exercise 4.6}

a) 01010110

b) 00110110

\subsubsection*{Exercise 4.7}

a) 01010110

b) 00110110

\subsubsection*{Exercise 4.8}

a) 10101010

b) 01101100

\subsubsection*{Exercise 4.9}

a)

\begin{itemize}
  \item 00111100
  \item 11000111
  \item 01010101
  \item 10000100
  \item 10101110
  \item 01111111
  \item 11101101
\end{itemize}

b)

\begin{itemize}
  \item 00111100
  \item 00001111
  \item 01010101
  \item 10000100
  \item 01100110
  \item 10110111
  \item 11101101
\end{itemize}

\subsubsection*{Exercise 4.10}

a)

(Row 1) Using disks 2, 3, and 5, recover disk 1
(Row 3) Using disks 1, 3, and 4, recover disk 7

b)

(Row 1) Using disks 2, 3, and 5, recover disk 1
(Row 2) Using disks 1, 2, and 6, recover disk 4

c)

(Row 1) Using disks 1, 2, and 5, recover disk 3
(Row 2) Using disks 1, 2, and 4, recover disk 6

\end{document}
