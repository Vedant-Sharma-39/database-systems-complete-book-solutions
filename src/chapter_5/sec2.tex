\documentclass[../../main.tex]{subfiles}

\begin{document}

\subsection{5.2 Extended Operators of Relational Algebra}

\subsubsection*{Exercise 2.1}

a)

$$
\pi_{A + B, A^2, B^2}(R) = \{(1, 0, 1), (5, 4, 9),
  (1, 0, 1), (6, 4, 16), (7, 9, 16)\}
$$

b)

$$
\pi_{B + 1, C - 1}(S) = \{(1, 0), (3, 3), (3, 4),
  (4, 3), (1, 1), (4, 3)\}
$$

c)

$$
\tau_{B, A}(R) = \{(0, 1), (0, 1), (2, 3),
  (2, 4), (3, 4)\}
$$

d)

$$
\tau_{B, C}(S) = \{(0, 1), (0, 2), (2, 4),
  (2, 5), (3, 4), (3, 4)\}
$$

e)

$$
\delta(R) = \{(0, 1), (2, 3), (2, 4), (3, 4)\}
$$

f)

$$
\delta(S) = \{(0, 1), (2, 4), (2, 5),
  (3, 4), (0, 2)\}
$$

g)

$$
\gamma_{A, \mathbf{SUM}(B)}(R) = \{(0, 2), (2, 7),
  (3, 4)\}
$$

h)

$$
\gamma_{B, \mathbf{AVG}(C)}(R) = \{(0, 1.5), (2, 4.5),
  (3, 4)\}
$$

i)

$$
\gamma_{A}(R) = \{(0), (2), (3)\}
$$

j)

\begin{align*}
  R \bowtie S &= \{(2, 3 , 4)\} \\
  \gamma_{A, \mathbf{MAX}(C)}(R \bowtie S) &= \{(2, 4)\}
\end{align*}

k)

$$
R \overset{\circ}{\bowtie}_{L} S = \{(0, 1, \bot),
  (0, 1, \bot), (2, 3, 4), (2, 4, \bot), (3, 4, \bot)\}
$$

l)

$$
R \overset{\circ}{\bowtie}_{R} S = \{(\bot, 0 , 1),
  (\bot, 2 , 4), (\bot, 2, 5), (2, 3, 4), (\bot, 0, 2),
  (\bot, 3, 4)\}
$$

m)

Combine above k) and l)

n)

\begin{align*}
  R \overset{\circ}{\bowtie}_{R.B < S.B} S = \{&(0, 1, 2, 4),
    (0, 1, 2, 5), (0, 1, 3, 4), (0, 1, 3, 4), \\
    &(0, 1, 2, 4), (0, 1, 2, 5), (0, 1, 3, 4), (0, 1, 3, 4), \\
    &(2, 3, \bot, \bot), (2, 4, \bot, \bot), (3 ,4, \bot, \bot), \\
    &(\bot, \bot, 0, 1), (\bot, \bot, 0, 2)\}
\end{align*}

\subsubsection*{Exercise 2.2}

a)

We use $R$ to represent the relation. For $\delta(R)$, there
is no duplicates. So for $\delta(\delta(R))$, the effect is none.
So $\delta(R) = \delta(\delta(R))$. And it is \emph{idempotent}

b)

It is \emph{idempotent}, because when we repeat the projection
it will yield the same relation.

c)

The result of $\sigma_{C}(R)$ is a relation where meets $C$.
Repeating the selection will yield the same results because
the relation already satisfy $C$. So it is \emph{idempotent}.



d)

The result is a relation whose schema consists of the grouping
attributes and the aggregated attributes. It is
\emph{not idempotent}.

e)

The result is sorted list of tuples based on some attributes $L$.
It is \emph{not idempotent}.

\subsubsection*{Exercise 2.3}

We could use classical relational algebra by the following operations.

\begin{align*}
  R1 &:= \rho_{A1, B1}(R) \\
  R2 &:= R1 \bowtie_{A = A1} R \\
  R3 &:= \pi_{A, A1}(R2)
\end{align*}

\end{document}
