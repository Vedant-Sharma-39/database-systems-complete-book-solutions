\documentclass[../../main.tex]{subfiles}

\begin{document}

\subsection{5.1 Relational Operations on Bags}

\subsubsection*{Exercise 1.1}

For simplicity, we assume that $R$ has $r$ tuples, $S$ has
$t$ tuples and $T$ has $t$ tuples.

a)

For a bag:

\begin{align*}
  \mbox{Tuples}((R \cup S) \cup T) &= (r + s) + t = r + s + t \\
  \mbox{Tuples}(R \cup (S \cup T)) &= r + (s + t) = r + s + t
\end{align*}

For a set:

Each set can have a tuple only once. If each set has a
common tuple then the result will have the single occurrence.

b)

\begin{align*}
  \mbox{Tuples}((R \cap S) \cap T) &= \min(\min(r, s), t) \\
  \mbox{Tuples}(R \cap (S \cap T)) &= \min(r, \min(s, t))
\end{align*}


Bags are essentially sets that allow the appearance of a
tuple more than once. In this case the operations on
a bag and set yield the same results.

c)

We let $R = \{a, b\}, S = \{b, c\}, T = \{c, d\}$.

\begin{align*}
  (R \bowtie S) \bowtie T &=
    (\{a, b\} \bowtie \{b, c\}) \bowtie \{c, d\} = \{a, b, c, d\} \\
  R \bowtie (S \bowtie T) &=
    \{a, b\} \bowtie (\{b, c\} \bowtie \{c, d\}) = \{a, b, c, d\}
\end{align*}

d)

For a bag:

Suppose a tuple $t$ occurs $n$ and $m$ times in $R$ and $S$
respectively. The union of these two bags in the bag $R \cup S$,
tuple $t$ would appear $n+m$ times. The union of these two bags
in the bag $S \cup R$ tuple $t$ would appear $m+n$ times.
Both sides of the relation yield the same result.

For a set:

In a set a tuple can only appear at most one time. Tuple $t$
might appear in set $R$ and $S$ $1$ or $0$ times. The
combinations of number of occurrences for tuple $t$
in $R$ and $S$ respectively are $(0,0), (0,1), (1,0)$,
and $(1,1)$. The union of either one of these
combinations on the right or left side of the relation
would yield the same result.

e)

For a bag:

Suppose a tuple $t$ occurs $n$ and $m$ times in $R$ and $S$
respectively. The intersection of these two bags in the bag $R \cap S$,
tuple $t$ would appear $\min(n, m)$ times. The intersection
of these two bags in the bag $S \cap R$ tuple $t$ would appear
$\min(n, m)$ times. Both sides of the relation yield the same result.

For a set:

In a set a tuple can only appear at most one time. Tuple $t$
might appear in set $R$ and $S$ $1$ or $0$ times. The
combinations of number of occurrences for tuple $t$
in $R$ and $S$ respectively are $(0,0), (0,1), (1,0)$,
and $(1,1)$. The intersection of either one of these
combinations on the right or left side of the relation
would yield the same result.

f)

We let $R = \{a, b\}, S = \{b, c\}$.

\begin{align*}
  R \bowtie S &= \{a, b\} \bowtie \{b, c\} = \{a, b, c\} \\
  S \bowtie R &= \{b, c\} \bowtie \{a, b\} = \{a, b, c\}
\end{align*}

g)

On left side we have projection and on the right we
have two projections. Union of the right side is
the same result as the left side. They have
same condition ($L$) to project.

h)

For a bag:

Let $R = \{x\}, S= \{x\}, T = \{x\}$

\begin{align*}
  R \cup (S \cap T) &=
    \{x\} \cup (\{x\} \cap \{x\}) = \{x, x\}\\
  (R \cup S) \cap (R \cup T) &=
    (\{x\} \cup \{x\}) \cap (\{x\} \cup \{x\}) = \{x, x\}
\end{align*}

For a set:

From set theory we can know that it holds for set.

i)

Does \textbf{not} hold for sets and bags. Left side when
the query executes can not be the same with the right side.
On right side we have two expressions and those two can
have some common tuples, nothing else. Let's take into
account that conditions $C$ and $D$ are not the same.

\subsubsection*{Exercise 1.2}

a)

Let $R = \{x\}, S= \{x, x\}, T = \{x\}$

\begin{align*}
  (R \cap S) - T) &=
    (\{x\} \cap \{x, x\}) - \{x\} = \emptyset\\
  R \cap (S - T) &=
    \{x\} \cap (\{x, x\} - \{x\}) = \{x\}
\end{align*}

b)

Let $R = \{x\}, S= \{x, x\}, T = \{x\}$

\begin{align*}
  R \cap (S \cup T) &=
    \{x\} \cap (\{x, x\} \cup \{x\}) = \{x\}\\
  (R \cap S) \cup (R \cap T) &=
    (\{x\} \cup \{x, x\}) \cap (\{x\} \cup \{x\}) = \{x, x\}
\end{align*}

c)

It holds for sets because \textbf{selection} on the left side
with condition OR will give the same result with the
\textbf{union} and situation on the right side.

Bags are unordered collection of elements \textbf{with elements}.
Because of that, bags behave differently. And this example does
\textbf{not} hold for bags.

\end{document}
