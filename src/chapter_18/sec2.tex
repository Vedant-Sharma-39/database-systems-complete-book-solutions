\documentclass[../../main.tex]{subfiles}

\begin{document}

\subsection{18.2 Conflict-Serializability}

\subsubsection*{Exercise 2.1}

a)

I don't give an example here, because the $t$ and $s$ is different,
thus $(T_{1}, T_{2})$ would be equivalent to $(T_{2}, T_{1})$.

b)

Need community help.

c)

According to the definition of serial schedules. The only serial order
that we have is $(T_{1}, T_{2})$ and $(T_{2}, T_{1})$. Thus we can

d)

To identify the number of serializable schedule of the 12 given actions,
we need to consider the interleaving of the serial order $(T_{1}, T_{2})$
and $(T_{2}, T_{1})$. And we have the following order:

\begin{itemize}
  \item $T_{1}$ operates on $A$ and $B$ before $T_{2}$.
  \item $T_{2}$ operates on $A$ and $B$ before $T_{1}$.
  \item $T_{1}$ operates on $A$ first, but $T_{2}$ operates on $B$ first.
\end{itemize}


$$
\left(\frac{6!}{3! \times (6 - 3)!}\right)^{2} = 400
$$

\subsubsection*{Exercise 2.2}

a)

We cannot swap the adjacent actions without a conflict, only
$(T_{1}, T_{2})$ is conflict equivalent to itself. So the answer is 1.

b)

We can swap the adjacent actions without a conflict, so the answer is

c)

$$
\left(\frac{4!}{2! \times (4 - 2)!}\right)^{2} = 36
$$

d)

The number of actions differs. This implies that we have different
answers as number of actions determines the number of possible
interleavings for the serializable order.

\subsubsection*{Exercise 2.3}

a)

$$
\frac{4!}{2! \times (4 - 2)!} \times 2 = 12
$$

b)

$$
\left(\frac{4!}{2! \times (4 - 2)!}\right)^{2} = 36
$$

\subsubsection*{Exercise 2.4}

Need community help.

\subsubsection*{Exercise 2.5}

Need community help.

\subsubsection*{Exercise 2.6}

Need community help.

\end{document}
